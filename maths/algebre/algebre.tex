
\subsection{Groupes, sous-groupes}
\begin{defi}
    Soient $(x,y)\in G^2$. $x*y*x^{-1}$ est le symétrique de $y$ sous l'action de $x$.
\end{defi}

\begin{prop}
    \[\prod_{i=1}^n G_i \text{ abélien } \equival\forall i \in \seg{1}{n}, G_i \text{ abélien}\]
\end{prop}

\begin{prop}\begin{itemize}
    \item Les sous-groupes de $(\mathbb{Z}, +)$ sont les groupes $(m\mathbb{Z}, +), m\in\mathbb{Z})$
    \item $\forall (a,b) \in \mathbb{Z}^2, a|b \Rightarrow b\mathbb{Z} \subset a\mathbb{Z}$
\end{itemize}\end{prop}

\begin{defi}[Sous-groupe engendré par]
    \[gr(X) = \bigcap_{\substack{H<G\\X\subset H}} H\]
\end{defi}

\begin{lemme}\[
    gr(X) = \{x_1* ... * x_n | n\in\mathbb{N}^*, (x_1, ...x_n) \in (X \cup X^{-1})^n \}
\]\end{lemme}

\begin{lemme} $\text{rang}(S_2) = 1 \text{ et } \forall n \in \llbracket 3, +\infty\llbracket, \text{rang}(S_n) = 2$
\end{lemme}

\begin{defi} Soit $G$ un groupe et $H < G$. $\forall (a,y) \in G^2, xR_Hy \equival x^{-1}.y \in H$. L'ensemble des classes d'équivalences est noté $G/H$.
\end{defi} 

\begin{prop} $xR_Hy \equival y \in xH$
\end{prop}

\begin{theo}[Théorème de Lagrange] $(G,.)$ groupe fini. On a :\[ |G| = |G/H| * |H| \]
\end{theo}

\begin{rem} \[\left\{\begin{array}{l} |G| = 2n \\ |H| = n \\ H < G \end{array}\right. \Rightarrow \forall g \in G, g^2 \in H \]
\end{rem}

\subsection{Morphismes de groupe}
\begin{prop}[Automorphismes de groupe] \begin{itemize}
    \item $G \text{ abélien} \Rightarrow (x \mapsto x^{-1}) \in Aut(G)$
    \item $\forall x \in G, \phi_x : y \mapsto x*y*x^{-1} \in Aut(G)$
    \item $Int(G) = {\phi_x | x \in G}$ l'ensemble des automorphismes intérieurs.
\end{itemize}\end{prop}

\begin{prop}[Morphismes de groupe] Soit $f : G \rightarrow G'$ un morphisme de groupe :
\begin{itemize}
    \item $H < G \Rightarrow f(H) < G'$
    \item $H < G' \Rightarrow f^{-1}(H) < G$
    \item $(Aut(G), \circ)$ est un groupe.
\end{itemize}
\end{prop}

\begin{defi} Soit $(G, *)$ un groupe et $R$ une relation d'équivalence sur $G$. $R$ est dite \emph{compatible} avec $*$ si :
    \[ \forall (x, y, z) \in G^3, xRy \Rightarrow \left\{\begin{array}{l} x*zRy*z \\ z*xRz*y\end{array}\right. \]
\end{defi}

\begin{prop} Soit $(G, *)$ un groupe et $H < G$. On a : \[ R_H \text{ compatible avec } * \equival \forall x \in G, (xH = Hx \equival xHx^{-1} = H \equival xHx^{-1} \subset H) \]
    On note alors $H \lhd G$. On dit que $H$ est \emph{distingué}/\emph{invariant} dans $G$.
\end{prop}

\begin{prop} Soit $H \lhd G$, alors $G/H$ possède un unique structure de groupe telle que \\$\Pi :\begin{array}{ccc} (G, *) & \rightarrow & (G/H, \Diamond) \\ x & \mapsto & \bar{x}\end{array}$ soit un morphisme de groupe.
\end{prop}

\begin{rem} \begin{itemize}
    \item Si $*$ commutative, alors $\Diamond$ commutative.
    \item Si $H \lhd G$, alors $(G/H, \Diamond)$ est appelé groupe quotient.
    \item Si $G$ abélien, alors $H \lhd G$.
\end{itemize}\end{rem}

\begin{prop} Soit $f:G\rightarrow G'$ morphisme de groupe. On a $Kerf \lhd G$.
\end{prop}

\begin{theo}[Premier théorème d'isomorphisme] $\bar{f} : \begin{array}{ccc} G/Kerf & \rightarrow & Imf \\ \bar{x} & \mapsto & f(x)\end{array}$ est bien définie et est un isomorphisme. On a alors, si $G$ est fini, $|G| = |Kerf| * |Imf|$.
\end{theo}

\subsection{Ordre d'un élément}
\begin{defi}[Ordre d'un élément] \[\omega(x) = \left\{\begin{array}{l} \min \{k\in\mathbb{N}^* | x^k = e\} \text{ si } \{k\in\mathbb{N}^* | x^k = e\}\neq\emptyset \\ +\infty\text{ sinon}\end{array}\right.\]
\end{defi}

\begin{prop} \[ \forall (x, y) \in G^2, \left\{\begin{array}{l}
            \omega(x) = \omega(x^{-1}) \\
            f\text{ injective }\Rightarrow \omega(f(x)) = \omega(x) \\
            \omega(xy) = \omega(yx)
    \end{array}\right.\]
\end{prop}

\begin{prop} $gr\{x\} = \omega(x)$ \end{prop}

\begin{prop} Si $G$ est un groupe fini, avec $|G| = n$, on a $\forall x \in G, \omega(x) | n$ \end{prop}

\begin{prop} Soit $G = gr\{x\}$. On a \begin{itemize}
    \item $\omega(x) = +\infty \Rightarrow (G,*)\simeq(\mathbb{Z},+)$
    \item $\omega(x) = n \Rightarrow (G,*)\simeq(\mathbb{Z}/n\mathbb{Z},+)$
\end{itemize}\end{prop}

\begin{prop}[Caractérisation de l'ordre d'un élément] Soit $G$ un groupe, $x\in G$ et $d\in\mathbb{N}^*$ :\begin{itemize}
    \item $\omega(x) = d \equival \forall k \in \mathbb{Z}, (x^k=e\equival d|k)$
    \item $\omega(x) < +\infty \Rightarrow \forall k \in \mathbb{Z}, \omega{x^k} = \frac{\omega(x)}{\omega(x)\wedge k}$
    \item $\omega(x) = +\infty \Rightarrow \forall k \in \mathbb{Z}, \omega(x^k) = +\infty$
\end{itemize}\end{prop}

\begin{prop} Soit $G=gr\{x\}$ fini de cardinal $n\in\mathbb{N}^*$. Soit $d\in\mathbb{N}$ tel que $d|n$. Il existe un unique sour groupe de $G$ de cardinal d, à savoir $gr\{x^{n/d}\}$.
\end{prop}

\subsection{Anneaux}
\begin{lemme} Soient $A$ un anneau. Soient $a$ et $b$ deux éléments de $A$ qui commutent. On a :\[\forall n \in \mathbb{N},\left\{\begin{array}{l}
            (a+b)^n = \sum_{k=0}^n {n\choose k} a^kb^{n-k} \\
            a^{n+1}-b^{n+1} = (a-b)\sum_{k=0}^n a^kb^{n-k} \\
\end{array}\right.\]\end{lemme}

\begin{defi} Soit $A$ un anneau. \begin{itemize}
    \item $x\in A$ est dit \emph{idempotent} si $x^2=x$.
    \item $x\in A$ est dit \emph{nilpotent} si $\exists k\in\mathbb{N}^*:x^k=0$.
    \item $A$ est dit intègre si $\forall (a,b)\in A^2, ab=0 \Rightarrow a = 0 ou b = 0$
\end{itemize}\end{defi}

\begin{prop} Soit $A$ un anneau.\begin{itemize}
    \item $A[X]$ intègre $\equival$ $A$ intègre.
    \item $A[[X]]$ intègre $\equival$ $A$ intègre.
\end{itemize}\end{prop}

\begin{prop} Soit $A$ un anneau.\begin{itemize}
    \item $A[X]^\times = A^\times$
    \item $A[[X]]^\times = \{S\in A[[X]] | \tilde{S}(0)\in A^\times\}$
\end{itemize}\end{prop}

\begin{prop} $\mathscr{Z}(\mathscr{M}_n(\mathbb{K})) = \{\lambda I_n | \lambda\in\mathbb{K}\}$
\end{prop}

\begin{lemme} Soit $f$ un morphisme d'anneaux. On a $f(A^\times)\in A'^\times$.
\end{lemme}

\subsection{Idéaux d'un anneau commutatif}
\begin{defi}[Idéal d'un anneau] Soit $(A,+,*)$ un anneau et $I\in\mathscr{P}(A)$. $I$ est un idéal si : \begin{itemize}
    \item $I<(A,+)$
    \item $\forall(a,x)\in A\times I, a*x\in I$
\end{itemize}\end{defi}

\begin{prop}\begin{itemize}
    \item Les seuls idéaux d'un corps sont les idéaux triviaux.
    \item Les idéaux de l'anneau $(\mathbb{Z}, +, *)$ sont les idéaux principaux $n\mathbb{Z}$.
    \item Les idéaux de $(\mathbb{K}[X], +, *)$ sont les idéaux principaux $P\mathbb{K}[X]$.
    \item Les idéaux de $(\mathbb{K}[[X]], +, *)$ sont les idéaux principaux $X^n\mathbb{K}[[X]]$.
\end{itemize}\end{prop}

\begin{prop}\begin{itemize}
    \item Soit $(I_k)_{k\in K}$ une famille d'idéaux. Alors $\cup_{k\in K} I_k$ est un idéal.
    \item Si $X$ est une partie de $A$, alors $id(X)$ est le plus petit idéal de $A$ contenant $X$.
    \item Si $I_1, \ldots I_n$ sont des idéaux de $A$, alors $id(I_1 \cap \ldots I_n) = I_1 + \ldots I_k$.
    \item $\forall (a_1, \ldots a_n)\in A^n, id\{a_1, \ldots a_n\} = a_1A + \ldots a_nA$
    \item Si $f$ est un morphisme d'anneaux, $\ker f$ est un idéal.
\end{itemize}\end{prop}

\begin{rem} Tout morphisme de corps est injectif.
\end{rem}

\begin{prop} On nomme $d$ le pgcd de $a_1$, \ldots $a_n$:
    \[\forall (a_1, \ldots a_n)\in\mathbb{Z}^n, \exists ! d\in\mathbb{Z}: \sum_{i=1}^na_i\mathbb{Z} = d\mathbb{Z}\]
\end{prop}

\begin{prop}[Anneau quotient] Soit $A$ un anneau et $I$ un idéal de $A$. Il existe une unique structure d'anneau commutatif unitaire sur $A/I$ telle que $\Pi : \begin{array}{ccc} A & \rightarrow & A/I \\ x & \mapsto & \bar{x} \end{array}$ soit un morphisme d'anneau. $A/I$ est alors appelé anneau quotient de $A$ par l'idéal $I$.
\end{prop}

\begin{theo}[Premier théorème d'isomorphisme] Si $f:A\rightarrow A'$ est un morphisme d'anneau, alors $\bar{f}:\begin{array}{ccc} A/\ker f & \rightarrow & Im\ f \\ \bar{x} & \mapsto & f(x)\end{array}$ est bein définie et est un isomorphisme d'anneau.
\end{theo}

\begin{defi}[Caractéristique] Posons $f:\begin{array}{ccc} \mathbb{Z} & \rightarrow & A \\ k & \mapsto & k.1_A \end{array}$. On a $\ker f = n\mathbb{Z}$. On nomme $car\ A = n$ la caractéristique de $A$ :\begin{itemize}
    \item Si $n \geq 1$, alors $\omega_{(A,+)}1_A = car\ A$.
    \item So $n = 0$, alors $\omega_{(A,+)}1_A = +\infty$.
\end{itemize}\end{defi}

\begin{prop}$\forall n \geq 2, \forall m \in \mathbb{Z}, m\wedge n = 1 \equival \mathbb{Z}/n\mathbb{Z} = gr\{\bar{m}\} \equival \bar{m}\in(\mathbb{Z}/n\mathbb{Z})^\times$
\end{prop}

\begin{prop}$\forall n\in\mathbb{N}^*, n\in\mathscr{P} \equival (\mathbb{Z}/n\mathbb{Z}, \bar{+}, \bar{*})\text{ est intègre} \equival(\mathbb{Z}/n\mathbb{Z}, \bar{+}, \bar{*}) \text{ est un corps}$
\end{prop}

\begin{prop}Pour $p\in\mathscr{P}$, il n'y a qu'un seul corps de cardinal $p$.
\end{prop}

\begin{prop}Si $\mathbb{K}$ est un corps fini, alors $car\ \mathbb{K}\in\mathscr{P}$
\end{prop}

