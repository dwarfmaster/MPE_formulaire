
\subsection{Groupes, sous-groupes}
\begin{defi}
    Soient $(x,y)\in G^2$. $x*y*x^{-1}$ est le symétrique de $y$ sous l'action de $x$.
\end{defi}

\begin{prop}
    \[\prod_{i=1}^n G_i \text{ abélien } \equival\forall i \in \seg{1}{n}, G_i \text{ abélien}\]
\end{prop}

\begin{prop}\begin{itemize}
    \item Les sous-groupes de $(\mathbb{Z}, +)$ sont les groupes $(m\mathbb{Z}, +), m\in\mathbb{Z})$
    \item $\forall (a,b) \in \mathbb{Z}^2, a|b \Rightarrow b\mathbb{Z} \subset a\mathbb{Z}$
\end{itemize}\end{prop}

\begin{defi}[Sous-groupe engendré par]
    \[gr(X) = \bigcap_{\substack{H<G\\X\subset H}} H\]
\end{defi}

\begin{lemme}\[
    gr(X) = \{x_1* ... * x_n | n\in\mathbb{N}^*, (x_1, ...x_n) \in (X \cup X^{-1})^n \}
\]\end{lemme}

\begin{lemme} $\text{rang}(S_2) = 1 \text{ et } \forall n \in \llbracket 3, +\infty\llbracket, \text{rang}(S_n) = 2$
\end{lemme}

\begin{defi} Soit $G$ un groupe et $H < G$. $\forall (a,y) \in G^2, xR_Hy \equival x^{-1}.y \in H$. L'ensemble des classes d'équivalences est noté $G/H$.
\end{defi} 

\begin{prop} $xR_Hy \equival y \in xH$
\end{prop}

\begin{theo}[Théorème de Lagrange] $(G,.)$ groupe fini. On a :\[ |G| = |G/H| * |H| \]
\end{theo}

\begin{rem} \[\left\{\begin{array}{l} |G| = 2n \\ |H| = n \\ H < G \end{array}\right. \Rightarrow \forall g \in G, g^2 \in H \]
\end{rem}

\subsection{Morphismes de groupe}
\begin{prop}[Automorphismes de groupe] \begin{itemize}
    \item $G \text{ abélien} \Rightarrow (x \mapsto x^{-1}) \in Aut(G)$
    \item $\forall x \in G, \phi_x : y \mapsto x*y*x^{-1} \in Aut(G)$
    \item $Int(G) = {\phi_x | x \in G}$ l'ensemble des automorphismes intérieurs.
\end{itemize}\end{prop}

\begin{prop}[Morphismes de groupe] Soit $f : G \rightarrow G'$ un morphisme de groupe :
\begin{itemize}
    \item $H < G \Rightarrow f(H) < G'$
    \item $H < G' \Rightarrow f^{-1}(H) < G$
    \item $(Aut(G), \circ)$ est un groupe.
\end{itemize}
\end{prop}

\begin{defi} Soit $(G, *)$ un groupe et $R$ une relation d'équivalence sur $G$. $R$ est dite \emph{compatible} avec $*$ si :
    \[ \forall (x, y, z) \in G^3, xRy \Rightarrow \left\{\begin{array}{l} x*zRy*z \\ z*xRz*y\end{array}\right. \]
\end{defi}

\begin{prop} Soit $(G, *)$ un groupe et $H < G$. On a : \[ R_H \text{ compatible avec } * \equival \forall x \in G, (xH = Hx \equival xHx^{-1} = H \equival xHx^{-1} \subset H) \]
    On note alors $H \lhd G$. On dit que $H$ est \emph{distingué}/\emph{invariant} dans $G$.
\end{prop}

\begin{prop} Soit $H \lhd G$, alors $G/H$ possède un unique structure de groupe telle que \\$\Pi :\begin{array}{ccc} (G, *) & \rightarrow & (G/H, \Diamond) \\ x & \mapsto & \bar{x}\end{array}$ soit un morphisme de groupe.
\end{prop}

\begin{rem} \begin{itemize}
    \item Si $*$ commutative, alors $\Diamond$ commutative.
    \item Si $H \lhd G$, alors $(G/H, \Diamond)$ est appelé groupe quotient.
    \item Si $G$ abélien, alors $H \lhd G$.
\end{itemize}\end{rem}

\begin{prop} Soit $f:G\rightarrow G'$ morphisme de groupe. On a $Kerf \lhd G$.
\end{prop}

\begin{theo}[Premier théorème d'isomorphisme] $\bar{f} : \begin{array}{ccc} G/Kerf & \rightarrow & Imf \\ \bar{x} & \mapsto & f(x)\end{array}$ est bien définie et est un isomorphisme. On a alors, si $G$ est fini, $|G| = |Kerf| * |Imf|$.
\end{theo}

\subsection{Ordre d'un élément}
